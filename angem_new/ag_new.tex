\documentclass[a4paper, 12pt]{article}
%\usepackage{mathtext}
\usepackage{cmap}
\usepackage[english, russian]{babel}
\usepackage[T2A]{fontenc}
\usepackage[utf8]{inputenc}
\usepackage[left=2cm, right=1.5cm, top=2cm, bottom=2cm]{geometry}
\usepackage{amsmath}
\usepackage{etoolbox}
\usepackage{amsthm}
\usepackage{amsfonts}
%\usepackage{indentfirst}
\usepackage{soul}
\usepackage{graphicx}
\usepackage{enumerate}
\usepackage{nicematrix}
\usepackage{booktabs}
\usepackage{mathtools,amssymb}

\usepackage{tikz,amstext}
\newlength{\tempheight}
\newcommand{\Let}[0]{%
	\mathbin{\text{\settoheight{\tempheight}{\mathstrut}\raisebox{0.5\pgflinewidth}{%
				\tikz[baseline,line cap=round,line join=round] \draw (0,0) --++ (0.4em,0) --++ (0,1.5ex) --++ (-0.4em,0);%
}}}}

\renewcommand{\phi}{\varphi}
\renewcommand{\epsilon}{\varepsilon}
\newcommand*\circled[1]{\tikz[baseline=(char.base)]{
            \node[shape=circle,draw,inner sep=2pt] (char) {#1};}}
\newcommand{\aug}{\fboxsep=-\fboxrule\!\!\!\fbox{\strut}\!\!\!}
\newcommand\tab[1][.5cm]{\hspace*{#1}}
\newcommand\undermat[2]{\makebox[0pt][l]{$\smash{\underbrace
			{\phantom{\begin{matrix}#2\end{matrix}}}_{\text{$#1$}}}$}#2}
\newcommand\overmat[2]{\makebox[0pt][l]{$\smash{\overbrace
			{\phantom{\begin{matrix}#2\end{matrix}}}^{\text{$#1$}}}$}#2}

\newcounter{lemcount}
\newcounter{lemcount2}
\newcounter{thcount}
\theoremstyle{definition}
\newtheorem*{definition}{Определение}
\newtheorem*{theorem}{Теорема}
\newtheorem*{consequense}{Следствие}
\newtheorem*{lemma}{Лемма}
\newtheorem*{subtheorem}{Утверждение}
\newtheorem*{formula}{Вывод формулы}
\newtheorem*{formulas}{Вывод формул}
\newtheorem*{remark}{Замечание}
\newtheorem*{examples}{Примеры}
\newtheorem*{example}{Пример}
\newtheorem*{lalala}{Упражнение}
\newtheorem*{algorithm}{Алгоритм}
\newtheorem*{properties}{Свойства}
\newtheorem*{properties1}{Свойство}
\newtheorem{lemmanum}[lemcount]{Лемма}
\newtheorem{lemmanum2}[lemcount2]{Лемма}
\newtheorem{theoremnum}[thcount]{Теорема}
% \newtheorem{theoremL}{Теорема}[section]
\usepackage[T2A]{fontenc}
\usepackage[utf8]{inputenc}
\usepackage[russian]{babel}
\addto\captionsenglish{% Replace "english" with the language you use
	\renewcommand{\contentsname}%
	{Содержание}%
}

\newsavebox{\boxedalignbox}
\newenvironment{boxedalign*}
  {\begin{equation*}\begin{lrbox}{\boxedalignbox}$\begin{aligned}}
  {\end{aligned}$\end{lrbox}\fbox{\usebox{\boxedalignbox}}\end{equation*}}

\usepackage{titlesec}
\titleformat{\section}{\LARGE \bfseries}{\thesection}{1em}{}
\titleformat{\subsection}{\Large\bfseries}{\thesubsection}{1em}{}
\titleformat{\subsubsection}{\large\bfseries}{\thesubsubsection}{1em}{}

\usepackage{hyperref}
\usepackage{xcolor}
% Цвета для гиперссылок
\definecolor{linkcolor}{HTML}{225ae2} % цвет ссылок
\definecolor{urlcolor}{HTML}{225ae2} % цвет гиперссылок
\hypersetup{
	pdfstartview=FitH, 
	linkcolor=linkcolor,
	urlcolor=urlcolor,
	colorlinks=true
}

\begin{document}

\section{Поверхности второго порядка}

\begin{definition}
    Поверхностью второго порядка называется множество точек трёхмерного аффинного или точечно-евклидова пространства, координаты которых в некоторой аффинной системе координат удовлетворяют уравнению $F(x, y, z) = 0$, где
    \[ F(x, y, z) = a_{11} x^2 + a_{22} y^2 + a_{33} z^2 + 2a_{12} xy + 2a_{13} xz + 2a_{23} yz + 2a_1 x + 2a_2 y + 2a_3 z + a_0 \]
    причём хотя бы одно из чисел $a_{11}, \ a_{22}, \ a_{33}, \ a_{12}, \ a_{13}, \ a_{23}$ отлично от нуля. Выражение $F(x, y, z)$ - \textit{многочлен второй степени} от переменных $x, y, z$.
    Уравнение $F(x, y, z) = 0$ называется \textit{общим уравнением} поверхности второго порядка.
\end{definition}

\begin{remark}
    Точно так же определяются повехности второго порядка в аффинном или точечно-евклидовом пространстве произвольной конечной размерности $n$;
    они задаются многочленами второй степени от $n$ переменных.
\end{remark}

Теория поверхностей второго порядка аналогична теории кривых второго порядка.

С каждым многочленом $F(x,y,z)$ связано \textit{квадратичное отображение} пространства (с данной системой координат) $f: A^3 \to \mathbb{R}$, которое каждой точке $X$ с координатами $(x, y, z)$ ставит в соответствие число $F(x,y,z)$.
Говорят, что это отображение представлено многочленом $F$ в данной системе координат.
В другой системе координат многочлен, представляющий ту же функцию, станет другим.

Как и в случае линий второго порядка:
\[ F(x,y,z) = \begin{pmatrix}
    x & y & z & 1
\end{pmatrix}
A
\begin{pmatrix}
    x \\ y \\ z \\ 1
\end{pmatrix}
=
\begin{pmatrix}
    x & y & z
\end{pmatrix}
A_1
\begin{pmatrix}
    x \\ y \\ z
\end{pmatrix} 
+ 2 \begin{pmatrix}
    x & y & z
\end{pmatrix}
\begin{pmatrix}
    a_1 \\ a_2 \\ a_3
\end{pmatrix} 
+ a_0,
\]
где
\[ A =
\begin{pmatrix}
    a_{11} & a_{12} & a_{13} & a_{1} \\
    a_{12} & a_{22} & a_{23} & a_{2} \\
    a_{13} & a_{23} & a_{33} & a_{3} \\
    a_{1} & a_{2} & a_{3} & a_{0} \\
\end{pmatrix}
\] – большая матрица,
\[ A_1 =
\begin{pmatrix}
    a_{11} & a_{12} & a_{13} \\
    a_{12} & a_{22} & a_{23} \\
    a_{13} & a_{23} & a_{33} \\
\end{pmatrix}
\] – малая матрица (квадратичной части).

\begin{definition}
    \[ F_1(x, y, z) = a_{11} x^2 + a_{22} y^2 + a_{33} z^2 + 2a_{12} xy + 2a_{13} xz + 2a_{23} yz\]
    называется \textit{квадратичной частью} многочлена $F$.
\end{definition}

Дословно так же, как в случае линий, доказывается, что при переходе к новой системе координат матрицы $A$ и $A_1$ многочлена $F$, представляющие всё ту же функцию $f: A^3 \to \mathbb{R}$, меняются по закону $A_1^{'} = C^T A_1 C$ и $A^{'} = D^T A D$, где $A_1^{'}$ и $A^{'}$ – матрицы в новых координатах, $C$ – матрица перехода от старого базиса к новому (её столбцы – координаты новых базисных векторов в старом базисе),
$D = \begin{pNiceMatrix}\Block{3-3}<\Huge>{C}&&&x_0\\&&&y_0\\&&&z_0\\0&0&0&1\end{pNiceMatrix}$, где
$x_0, y_0, z_0$ - координата нового начала координат в старой системе координат.

В новой системе координат:
\[ F^{'}(x^{'},y^{'},z^{'}) = \begin{pmatrix}
    x^{'} & y^{'} & z^{'} & 1
\end{pmatrix}
A^{'}
\begin{pmatrix}
    x^{'} \\ y^{'} \\ z^{'} \\ 1
\end{pmatrix}
+ 2 \begin{pmatrix}
    x^{'} & y^{'} & z^{'}
\end{pmatrix}
A_1^{'}
\begin{pmatrix}
    a_1^{'} \\ a_2^{'} \\ a_3^{'}
\end{pmatrix} 
+ a_0,
\]
где буквы со штрихами – координаты, многочлен и матрицы в новой системе координат,

\[
\begin{pmatrix}
    a_1^{'} \\ a_2^{'} \\ a_3^{'}
\end{pmatrix} = C^T 
\begin{pmatrix}
    a_1 \\ a_2 \\ a_3
\end{pmatrix}.
\]

Будем считать, что дело происходит в точечно-евклидовом пространстве $\mathbb{R}^3$ и многочлен $F$, представляющий квадратичное отображение $f: \mathbb{R}^3 \to \mathbb{R}$, задан в прямоугольной системе координат. Это не умаляет общности: если задана не прямоугольная система координат, то мы всегда можем перейти в прямоугольную (по выписанным выше формулам), а если дело происходит в аффинном пространстве, то мы можем временно превратить его в евклидово, определив скалярное произведение в данной системе координат (в которой записана функция $F(x,y,z)$) по формуле $\left(
     \begin{pmatrix}
        x_1 \\ y_1 \\ z_1
     \end{pmatrix},
     \begin{pmatrix}
        x_2 \\ y_2 \\ z_2
     \end{pmatrix}
\right)
= x_1 x_2 + y_1 y_2 + z_1 z_2$.
Эта формула действительно задаёт некоторое скалярное произведение, и наша система координат прямоугольна относительно него.

Совершенно так же (и из тех же соображений), как в случае линий, мы можем найти каноническую систему координат (прямоугольную!), в которой уравнение поверхности имеет простейший вид:

\begin{enumerate}
    \item Решаем характеристическое уравнение $|A - \lambda E| = 0$, находим корни $\lambda_1, \lambda_2, \lambda_3$ - характеристические числа (собственные значения).
    
    В этом месте отличие (от случая линии): если нулевых корней $\leqslant 1$, то первые номера даём положительным $\lambda$ (упорядочиваем по возрастанию), следующие – отрицательным (по убыванию), потом идёт 0 (если есть). Если получилось два нулевых корня, то $\lambda_1 = \lambda_3 = 0, \ \lambda_2 \neq 0$.
    \item Для каждого $i = 1, 2, 3$ решаем однородную систему уравнений $(A - \lambda_i E) \begin{pmatrix} x \\ y \\ z \end{pmatrix} = \begin{pmatrix} 0\\0\\0 \end{pmatrix}$, находим ненулевое решение (оно обязательно существует, так как $|A - \lambda_i E| = 0$). Если есть несколько линейно независимых решений (два или три), выбираем максимально возможное число линейно независимых решений так, чтобы они были взаимно ортогональны;
    другими словами, выбираем ортогональный базис в пространстве решений (мы знаем, что это возможно, поскольку какой-то базис есть всегда, а ортогонализировать базисы мы уже научились).
    \item Нормируем полученные на предыдущем шаге решения (собственные векторы, соответствующие собственным значениям $\lambda_1, \lambda_2, \lambda_3$):
    если $\overline{u_i}$ – решение, соответствующее числу $\lambda_i$, то полагаем $\overline{e_i^{'}} = \frac{\overline{u_i}}{|\overline{u_i}|}$.
    Векторы $\overline{e_1^{'}}, \overline{e_2^{'}}, \overline{e_3^{'}}$ будут базисными векторами канонической системы координат. По построение они образуют ортонормированный базис.
    \item Избавляемся от линейной части, насколько возможно.
    Выделяя полные квадраты и меняя начало координат: в новом базисе матрица $A_1^{'}$ диагональна, и если в выражении $F^{'}(x^{'},y^{'},z^{'})$ есть, скажем, $a_{11}^{'} x^{'2} + 2a_{1}^{'} x^{'}$, то меняем $x^{'}$ на $x^{''} + x_0^{'}$, где $x_0^{'}$ – число, первая координата нового начала координат в системе координат с новым базисом) так, чтобы было
    $a_{11}^{'} x^{'2} + 2a_{1}^{'} x^{'} = a_{11}^{'} x^{''2} + c_1$, где $c_1$ – константа.
    \item Если удалось избавиться от всех линейных членов, то мы получили канонический вид уравнения, а заодно и каноническую систему координат: её базис – вектора $\overline{e_1^{'}}, \overline{e_2^{'}}, \overline{e_3^{'}}$, найденные на третьем шаге, а начало – точка $O^{'}$ с координатами $x_0^{'}, y_0^{'}, z_0^{'}$ (относительно системы координат со старым началом и новым базисом), найденным на четвёртом шаге.
    Если не удалось избавиться от, например, $a_3^{'} z^{'}$, то есть в квадратичной части $F^{'}(x^{'},y^{'},z^{'})$ нет члена $a_33^{'} z^{'2}$ (это означает, что $\lambda_3 = 0$), но от других линейных членов избавиться удалось, то сдвигаем начало координат по оси $O^{'} z^{'}$ так, чтобы избавиться от всех накопившихся при избавлении от других линейных членов констант:
    $z'' = z''' - \frac{c_1 + c_2 + a_0}{a'_3}$. Ничего себе, оказывается кавычку можно было не экранировать... Надо теперь всё переделывать(((
    Третьей координатой нового начала координат (в системе координат с новым базисом и старым началом) будет $- \frac{c_1 + c_2 + a_0}{a'_3}$, а первыми двумя координатами будут $x'_0$ и $y'_0$, найденные на шаге 4. Если не удалось избавиться от линейных членов с $x$ и $z$ (и тогда $\lambda_1 = \lambda_3 = 0$), то на четвёртом шаге получилось уравнение $\lambda_2 y''^2 + 2a''_1 x'' + 2a''_3 z'' + c_2 + a_0 = 0$, где $c_2$ – константа, которая вылезла при избавлении от $2a'_2 y'$.
    В этом случае ещё раз меняем базис:
    $\overline{e''_1} = \left( \frac{a''_1}{\sqrt{a_1^{''2} + a_3^{''2}}}, 0, \frac{a''_3}{\sqrt{a_1^{''2} + a_3^{''2}}} \right)$,
    $\overline{e''_2} = \overline{e'_2}$,
    $\overline{e''_3} = \left( \frac{a''_3}{\sqrt{a_1^{''2} + a_3^{''2}}}, 0, -\frac{a''_3}{\sqrt{a_1^{''2} + a_3^{''2}}} \right)$.
    Новый базис по-прежнему ортогональный, и после перехода к нему уравнение примет вид $\lambda_2 y'''^2 + 2a'''_1 x''' + c = 0$.
    Остаётся избавиться от константы сдвигом по оси $x'''$.
    \item В результате получим почти каноническую систему координат и \textit{простейшее уравнение} поверхности. Его надо будет поделить на число (и, возможно, поменять направление некоторых базисных векторов и(или) поменять местами некоторые базисные векторы), чтобы получилось уравнение одно из семнадцати видов, перечисленных далее.
    Например, если получилось $y^2 = -2px$, для $p > 0$, надо поменять местами направления вектора $\overline{e''_1}$ (умножить его на -1), а если получилось
    $\frac{x^2}{a^2} - \frac{y^2}{b^2} = -1$, надо поменять местами $\overline{e''_1}$ и $\overline{e''_2}$.
\end{enumerate}

Если есть цель сохранить ориентацию системы координат, то может понадобиться ещё один шаг: надо посмотреть, какой определитель у произведения всех матриц перехода (т.е. у матрицы перехода от самого первого к самому последнему базису) и если он $-1$, поменять направление (умножить на $-1$) вектор $\overline{e''_2}$ (линейных членов с $y$ в каноническом уравнении не бывает, так что это ничего не испортит).

\end{document}