\documentclass[a4paper]{article}

\usepackage{cmap}
\usepackage[T2A]{fontenc}
\usepackage[english, russian]{babel}
\usepackage[utf8]{inputenc}
\usepackage[left=2cm,right=1.5cm,top=2cm,bottom=2cm]{geometry}
\usepackage{amsmath}
\usepackage{amssymb}
\usepackage{etoolbox}
\usepackage{amsthm}
\usepackage{booktabs}
\usepackage{graphicx}

\graphicspath{{./calculus_definition/}}

\newcommand{\Z}{\mathbb{Z}}
\newcommand{\N}{\mathbb{N}}
\newcommand{\R}{\mathbb{R}}
\newcommand{\Q}{\mathbb{Q}}
\newcommand{\B}{\mathfrak{B}}
\newcommand{\Bo}{\mathring{B}}
\renewcommand{\phi}{\varphi}
\renewcommand{\epsilon}{\varepsilon}
\renewcommand{\emptyset}{\varnothing}
\newcommand{\lra}{\Leftrightarrow}
\newcommand{\xn}{\{x_n\}_{n = 1}^{\infty}}
\renewcommand{\div}{\mid}
\newcommand{\ndiv}{\nmid}
\newcommand{\om}{\bar{\bar{o}}}
\newcommand\tab[1][.5cm]{\hspace*{#1}}

\newcommand{\lims}{\lim\limits_{n\to \infty}}
\renewcommand{\liminf}{\lim\limits_{x\to \infty}}
\newcommand{\liminfp}{\lim\limits_{x\to +\infty}}
\newcommand{\liminfm}{\lim\limits_{x\to -\infty}}
\newcommand{\lima}{\lim\limits_{x\to a}}
\newcommand{\limo}{\lim\limits_{x\to 0}}

\newcounter{concount}

\theoremstyle{plain}
\newtheorem{theorem}{Теорема}[section]
\newtheorem{corollary}{Следствие}[theorem]
\theoremstyle{definition}
\newtheorem{definition}{Определение}[section]
\newtheorem{example}{Пример}[section]
\theoremstyle{remark}
\newtheorem{remark}{Замечание}[section]

\usepackage{titlesec}
\titleformat{\section}{\LARGE \bfseries}{\thesection}{1em}{}
\titleformat{\subsection}{\Large\bfseries}{\thesubsection}{1em}{}
\titleformat{\subsubsection}{\large\bfseries}{\thesubsubsection}{1em}{}

\usepackage{hyperref}
\usepackage{xcolor}
\definecolor{linkcolor}{HTML}{225ae2}
\definecolor{urlcolor}{HTML}{225ae2}
\hypersetup{
    pdfstartview=FitH, 
    linkcolor=linkcolor,
    urlcolor=urlcolor,
    colorlinks=true
}

\title{Стандарты ведения конспектов ЧЕРНОВАЯ ВЕРСИЯ}
\author{Цыбулин Егор}
\date{\today}

\begin{document}
    \maketitle
    \section{Основные положения}
    \begin{enumerate}
        \item Рукописный конспект (далее Конспект) - это заметка, написанная от руки на бумаге, имеющая какие-либо теоретические (реже - практические) знания.
        \item Конспект оформляется на белых листах A4 с минимальной плотностью листа: 80г$/$м$^2$ с четырьмя отверстиями для возможного использования Конспекта в тетрадях со вставными листами.
        \item Для конспектов используются чёрная и синяя ручки 0,5мм (в 2024 году крайне желательно использовать Flair Writometer). Дополнительно можно использовать текстовыделители пастельных оттенков для выделения параграфов или разделения лекций.
    \end{enumerate}


    \section{Концепция правильного ведения Конспекта}
    \begin{enumerate}
        \item Конспект должен быть легко читаемым.
        \item Конспект должен быть легко сканируемым.
        \item Конспект должен быть легко переводимым в электронный формат.
        \item Конспект должен иметь возможность удобно использоваться в различных системах организации информации.
        \item Конспект должен быть строгим и лаконичным.
    \end{enumerate}


    \section{Форматирование бумаги}
    \begin{enumerate}
        \item Отступ слева должен быть достаточным, чтобы использовать листы в совокупности, например в скоросшивателе.
        \item Отступ справа должен быть чуть меньше отступа слева, достаточный для того, чтобы было удобно держать Конспект в руках.
        \item Отступ сверху должен быть около 1-1,5 сантиметров для выделения заголовка от остального текста, междустрочный интервал которого, очевидно, меньше 1,5 сантиметров.
        \item Междустрочный интервал основного текста должен быть в районе 1,25-1,5.
    \end{enumerate}


    \section{Форматирование заголовочного текста}
    \begin{enumerate}
        \item Заголовок должен быть написан буквами большего, чем основного текста, размера. Крайне желательно использование печатных букв в заголовке. Также заголовок пишется чёрной ручкой.
        \item Для лекций обязательно указывать дату недалеко от разделителя лекций, в котором должно быть написано "Лекция №" и номер соответствующей лекции. Слово "Лекция" должно быть подчёркнуто одной чертой. Для всех остальных Конспектов использование даты остаётся на усмотрение автора.
        \item Новые параграфы, разделы и т.д. пишутся чёрной ручкой с заглавной буквы. Подчёркивание необязательно.
    \end{enumerate}

    
    \section{Форматирование основного текста}
    \begin{enumerate}
        \item Каждый отдельный теоретический блок отмечается жирной точкой в начале строки.
        \item Дочерние теоретические блоки должны быть на один пункт правее материнского блока.
        \item Теоремы, следствия, леммы и т.д. должны быть написаны чёрной ручкой с заглавной буквы. Соответствующее слово должно быть подчёркнуто одной чертой снизу, после него должно стоять двоеточие. Если теорема, лемма и т.д. именные, то имя пишется в скобках до двоеточия и также подчёркивается одной чертой снизу. Формулировки пишутся также чёрной ручкой.
        \item Замечания должны соответствовать такому же форматированию, как и более важные фрагменты теории, которые были описаны пунктом выше, но само замечание должно быть написано, напротив, синей ручкой.
        \item Примеры должны соответствовать такому же форматированию, как и более важные фрагменты теории, которые были описаны пунктом выше, но слово "пример" должно быть с маленькой буквы. Сами примеры должны быть написаны, напротив, синей ручкой.
        \item Доказательства должны соответствовать такому же форматированию, как и примеры. Специфика их использования предполагает связь с теоремой, леммой, следствием и т.д., следовательно, они должны быть на один пункт правее соответствующего теоретического блока. Доказательства должны быть написаны синей ручкой. После доказательства ставится символ Халмоша.
        \item Определения оформляются следующим образом: пишется слово "опр." чёрной ручкой с маленькой буквы, подчёркивается, после ставится двоеточие. Непосредственно определение пишется синей ручкой. Допускается использование восклицательного знака вместо слова "опр.", однако тогда необходимо соблюдать единообразие оформления определений во всех записях из данного Конспекта.
        \item После важного теоретического блока должен быть выдержан некоторый отступ.
    \end{enumerate}



\end{document}
