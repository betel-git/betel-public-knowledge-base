\documentclass[a4paper]{article}

\usepackage{cmap}
\usepackage[T2A]{fontenc}
\usepackage[english, russian]{babel}
\usepackage[utf8]{inputenc}
\usepackage[left=2cm,right=1.5cm,top=2cm,bottom=2cm]{geometry}
\usepackage{amsmath}
\usepackage{amssymb}
\usepackage{etoolbox}
\usepackage{amsthm}
\usepackage{booktabs}
\usepackage{graphicx}

\graphicspath{{./calculus_definition/}}

\newcommand{\Z}{\mathbb{Z}}
\newcommand{\N}{\mathbb{N}}
\newcommand{\R}{\mathbb{R}}
\newcommand{\Q}{\mathbb{Q}}
\newcommand{\B}{\mathfrak{B}}
\newcommand{\Bo}{\mathring{B}}
\renewcommand{\phi}{\varphi}
\renewcommand{\epsilon}{\varepsilon}
\renewcommand{\emptyset}{\varnothing}
\newcommand{\lra}{\Leftrightarrow}
\newcommand{\xn}{\{x_n\}_{n = 1}^{\infty}}
\renewcommand{\div}{\mid}
\newcommand{\ndiv}{\nmid}
\newcommand{\om}{\bar{\bar{o}}}
\newcommand\tab[1][.5cm]{\hspace*{#1}}

\newcommand{\lims}{\lim\limits_{n\to \infty}}
\renewcommand{\liminf}{\lim\limits_{x\to \infty}}
\newcommand{\liminfp}{\lim\limits_{x\to +\infty}}
\newcommand{\liminfm}{\lim\limits_{x\to -\infty}}
\newcommand{\lima}{\lim\limits_{x\to a}}
\newcommand{\limo}{\lim\limits_{x\to 0}}

\newcounter{concount}

\theoremstyle{plain}
\newtheorem{theorem}{Теорема}[section]
\newtheorem{corollary}{Следствие}[theorem]
\newtheorem*{st}{Формулировка}
\theoremstyle{definition}
\newtheorem{definition}{Определение}[section]
\newtheorem{example}{Пример}[section]
\theoremstyle{remark}
\newtheorem{remark}{Замечание}[section]

\usepackage{titlesec}
\titleformat{\section}{\LARGE \bfseries}{Билет \thesection}{1em}{}
\titleformat{\subsection}{\Large\bfseries}{\thesubsection}{1em}{}
\titleformat{\subsubsection}{\large\bfseries}{\thesubsubsection}{1em}{}

\usepackage{hyperref}
\usepackage{xcolor}
\definecolor{linkcolor}{HTML}{225ae2}
\definecolor{urlcolor}{HTML}{225ae2}
\hypersetup{
    pdfstartview=FitH, 
    linkcolor=linkcolor,
    urlcolor=urlcolor,
    colorlinks=true
}

\title{Программа экзамена по математическому анализу}
\author{Цыбулин Егор}
\date{\today}

\begin{document}
\maketitle
%\tableofcontents
\section{}
\begin{st}
    Множества, кванторы, подмножества. Основные операции на множествах и их свойства. Прямое произведение множеств.
\end{st}

\textbf{Определения:}
\begin{enumerate}
    \item Подмножество
    \item Пустое множество
    \item Пересечение множеств
    \item Объединение множеств
    \item Разность множеств
    \item Одноэлементное множество
    \item Пара
    \item Упорядоченная пара
    \item Декартово (прямое) произведение
\end{enumerate}


\section{}
\begin{st}
    Отображения, классификация отображений, области определния и значений. Образы и прообразы множеств при отображениях, обратное отображение.
\end{st}

\textbf{Определения:}
\begin{enumerate}
    \item Отображение (функция)
    \item Область определения функции
    \item Область значения функции
    \item Инъекция
    \item Сюръекция
    \item Биекция
    \item Ограничение $f$ на $X_1$
    \item Образ множества
    \item (Полный) прообраз множества
    \item Обратное отображение
    \item Правила де Моргана
\end{enumerate}


\section{}
\begin{st}
    Аксиоматика Пеано натурального ряда. Конечные множества. Отношение порядка. Порядок на $\N$ (без доказательства). Операции сложения и умножения.
\end{st}

\textbf{Определения:}
\begin{enumerate}
    \item 4 аксиомы Пеано
    \item n-элементное множество
    \item Конечное множество
    \item Бесконечное множество
    \item Отношение
    \item Отношение порядка
    \item Арифметические операции (сумма, разность, произведение, частное)
\end{enumerate}

\textbf{Теоремы:}
\begin{enumerate}
    \item О единственности отношения порядка на $\N$ (без доказательства)
    \item Принцип наименьшего элемента
\end{enumerate}


\section{}
\begin{st}
    Целые числа, их свойства. Рациональные числа, их свойства. Аксиоматика архимедова упорядоченного числового поля.
\end{st}

\textbf{Определения:}
\begin{enumerate}
    \item Множество целых чисел
    \item 6 свойств целых чисел
    \item Множество рациональных чисел
    \item 14 свойств рациональных чисел
    \item Упорядоченное поле
    \item Архимедово поле
\end{enumerate}


\section{}
\begin{st}
    Аксиома полноты, действительные числа. Полнота модели бесконечных десятичных дробей. Модель действительных чисел как числовой прямой, модель действительных чисел как множества сечений рациональных чисел.
\end{st}

\textbf{Определения:}
\begin{enumerate}
    \item Множество действительных чисел
    \item Аксиома полноты
    \item Последовательность
    \item Бесконечная десятичная дробь (БДД)
    \item Отношение порядка для БДД
    \item Дедекиндовы сечения
    \item Геометрическая модель числовой прямой
\end{enumerate}

\textbf{Теоремы:}
\begin{enumerate}
    \item Теорема о том, что модель БДД удовлетворяет аксиоме полноты
    \item Теорема о том, что на множестве всех пар сечений можно ввести операции и отношение таким образом, что будут выполняться все 16 аксиом
\end{enumerate}


\section{}
\begin{st}
    Ограниченные множества в $\R$, точные грани. Принцип полноты Вейерштрасса. Промежутки действительных чисел. Принцип полноты Кантора.
\end{st}

\textbf{Определения:}
\begin{enumerate}
    \item Максимальный (минимальный) элемент
    \item Верхняя (нижняя) грань
    \item Ограниченное (сверху, снизу) множество
    \item Точная верхняя (нижняя) грань
    \item Промежутки
    \item Модуль 
\end{enumerate}

\textbf{Теоремы:}
\begin{enumerate}
    \item Принцип полноты Вейерштрасса
    \item Свойство точной верхней грани
    \item Принцип вложенных отрезков (принцип полноты Кантора)
\end{enumerate}


\section{}
\begin{st}
    Отношение эквивалентности. Равномощность множеств. Равномощность как отношение эквивалентности.
\end{st}

\textbf{Определения:}
\begin{enumerate}
    \item Отношение эквивалентности
    \item Равномощность
\end{enumerate}

\textbf{Теоремы:}
\begin{enumerate}
    \item Теорема о том, что равномощность множеств является отношением эквивалентности
    \item Теорема о том, что конечные множества равномощны тогда и только тогда, когда они содержат одинаковое количество элементов
\end{enumerate}


\section{}
\begin{st}
    Счётные множества, их свойства, примеры. Несчётность интервала. Множества мощности континуум. Сравнение мощностей как отношение порядка (без доказательства).
\end{st}

\textbf{Определения:}
\begin{enumerate}
    \item Счётное множество
    \item Не более чем счётное множество
    \item Примеры счётных множеств
    \item Множество мощности континуум
\end{enumerate}

\textbf{Теоремы:}
\begin{enumerate}
    \item Теорема о том, что объединение не более счётного числа счётных множеств счётно
    \item Теорема о том, что объединение не более чем счётного числа не более чем счётных множеств не более чем счётно (без доказательства)
    \item Теорема Кантора (несчётность интервала)
    \item Следствие о том, что действительных чисел несчётно (множество мощности континуум)
    \item Теорема о том, что у любого множества мощность множества всех подмножеств строго больше, чем мощность самого множества (без доказательства)
    \item Теорема о том, что сравнение мощностей является отношением порядка
    \item Теорема о том, что у любого бесконечного множества существует счётное подмножество
    \item Теорема о том, что если А - бесконечное, B - не более чем счётное, то $A \sim A \cup B $
\end{enumerate}


\section{}
\begin{st}
    Окрестности точки. Классификация точек относительно подмножеств действительных чисел. Открытые и замкнутые множества, их свойства.
\end{st}

\textbf{Определения:}
\begin{enumerate}
    \item $\epsilon$-окрестность
    \item Проколотая $\epsilon$-окрестность
    \item Внутренняя точка множества
    \item Внешняя точка множества
    \item Граничная точка множества
    \item Внутренность множествва
    \item Внешность множества
    \item Граница множества
    \item Предельная точка
    \item Изолированная точка
    \item Точка прикосновения
    \item Множество Кантора
    \item Открытое множество
    \item Замкнутое множество
\end{enumerate}

\textbf{Теоремы:}
\begin{enumerate}
    \item Утверждение о том, что точки прикосновения множества являются либо внутренними, либо граничными
    \item Утверждение о том, что точки прикосновения множества являются либо предельными, либо изолированными
    \item Теорема с объединениями / пересечениями открытых / замкнутных множеств
\end{enumerate}


\section{}
\begin{st}
    Критерии замкнутости множеств. Свойства замкнутых множеств. Компакты. Компактность отрезка. Теорема Больцано-Вейерштрасса.
\end{st}

\textbf{Определения:}
\begin{enumerate}
    \item Покрытие
    \item Компакт
\end{enumerate}

\textbf{Теоремы:}
\begin{enumerate}
    \item Теорема-критерий замкнутости множеств
    \item Теорема о том, что если $A$ - ограничено сверху или снизу и замкнуто, то существует $\max{A}$ или $\min{A}$, соответственно
    \item Теорема о том, что любой отрезок является компактом
    \item Лемма Гейне-Бореля (без доказательства)
    \item Теорема Больцано-Вейерштрасса
\end{enumerate}


\section{}
\begin{st}
    Числовые последовательности, подпоследовательности, предел. Свойства последовательностей, имеющих предел.
\end{st}

\textbf{Определения:}
\begin{enumerate}
    \item Последовательность
    \item Ограниченность последовательности
    \item Подпоследовательность
    \item Предел последовательности
\end{enumerate}

\textbf{Теоремы:}
\begin{enumerate}
    \item Теорема о том, что если последовательность сходится, то её предел единственный
    \item Теорема о том, что если последовательность имеет предел, то и любая её подпоследовательность имеет тот же предел
    \item Теорема об отделимости
\end{enumerate}


\section{}
\begin{st}
    О-символика. Бесконечно малые и бесконечно большие последовательности. Свойства бесконечно малых и бесконечно больших последовательностей. Арифметические свойства сходящихся последовательностей. Критерий Коши сходимости последовательностей. Примеры.
\end{st}

\textbf{Определения:}
\begin{enumerate}
    \item О-малое
    \item О-большое
    \item Бесконечно малая последовательность
    \item Бесконечно большая последовательность
    \item Фундаментальная последовательность
\end{enumerate}

\textbf{Теоремы:}
\begin{enumerate}
    \item Арифметические свойства бесконечно малых (как теорема)
    \item Теорема о том, что если $a_n$ - бесконечно большая и $a_n \neq 0$, то $\frac{1}{a_n}$ - бесконечно малая
    \item Лемма $\liminf{a_n} = a \Leftrightarrow a_n - a = \om(1)$
    \item Теорема том, что если последовательность ограничена, то существует подпоследовательность, которая стремится к пределу
    \item Арифметические свойства сходящихся последовательностей
    \item Критерий Коши сходимости последовательностей
\end{enumerate}


\section{}
\begin{st}
    Предельный переход в неравенствах. Неравенство Бернулли. Бином Ньютона.
\end{st}

\textbf{Теоремы:}
\begin{enumerate}
    \item Предельный переход в неравенствах и следствие
    \item Следствие из предельного перехода
    \item Теорема о двух милиционерах
    \item Неравенство Бернулли
    \item Бином Ньютона
\end{enumerate}


\section{}
\begin{st}
    Монотонные последовательности, их свойства. Число "е".
\end{st}

\textbf{Определения:}
\begin{enumerate}
    \item Монотонные последовательности
\end{enumerate}

\textbf{Теоремы:}
\begin{enumerate}
    \item Теорема о том, что если последовательность монотонна и ограничена, то у неё есть предел
    \item Число "е" и всё, что с ним связано в рамках последовательностей
\end{enumerate}


\section{}
\begin{st}
    Частичные пределы последовательностей, их свойства. Верхний и нижний пределы последовательностей, их свойства.
\end{st}

\textbf{Определения:}
\begin{enumerate}
    \item Частичный предел
    \item Верхний (нижний) предел
\end{enumerate}

\textbf{Теоремы:}
\begin{enumerate}
    \item Теорема о связи частичных пределов и замкнутости множества
    \item Теорема о верхних и нижних пределах
    \item Критерий сходимости в терминах частичных пределов
\end{enumerate}


\section{}
\begin{st}
    Предел функции по Коши и по Гейне, их эквивалентность. Основные свойства предела функции.
\end{st}

\textbf{Опредлеения:}
\begin{enumerate}
    \item Предел по Коши
    \item Предел по Гейне
    \item Предел функции в точке по множеству
    \item Односторонние пределы
\end{enumerate}

\textbf{Теоремы:}
\begin{enumerate}
    \item Теорема об эквивалентности определений
    \item Теорема о том, что если у функции существует предел в точке, то он единственный
    \item Теорема об ограниченности сходящейся последовательности в некоторой проколотой $\delta$-окрестности
    \item Теорема об отделимости для функций
    \item Критерий существования предела в точке в терминах односторонних пределов
\end{enumerate}


\section{}
\begin{st}
    О-символика для функций, бесконечно малые и бесконечно большие функции. Исчисление бесконечно малых, арифметические свойства предела. Предельный переход в неравенствах. Критерий Коши существования предела функции.
\end{st}

\textbf{Определения:}
\begin{enumerate}
    \item О-малое для функции
    \item Бесконечно малая функция
    \item О-большое для функции
    \item Бесконечно большая функция
\end{enumerate}

\textbf{Теоремы:}
\begin{enumerate}
    \item Исчисление бесконечно малых (арифметические свойства как теорема)
    \item Теорема о том, что предел функции $f(x)$ $a$ в точке $x_0$ существует тогда и только тогда, когда $f(x) = a + \om (1), \ x \to x_0$
    \item Арифметические свойства пределов функций
    \item Предельный переход в неравенствах
    \item Следствие из предельного перехода
    \item Теорема о двух милиционерах
    \item Критерий Коши для функций
\end{enumerate}


\section{}
\begin{st}
    Монотонные функции, теорема о пределе монотонной и ограниченной функции.
\end{st}

\textbf{Определения:}
\begin{enumerate}
    \item Монотонные функции
\end{enumerate}

\textbf{Теоремы:}
\begin{enumerate}
    \item Теорема о пределе монотонной и ограниченной функции
\end{enumerate}


\section{}
\begin{st}
    Непрерывные функции, локальные свойства непрерывных функций. Точки разрыва и их классификация.
\end{st}

\textbf{Определения:}
\begin{enumerate}
    \item Непрерывность функции в точке
    \item Точки устранимого разрыва
    \item Точки разрыва первого рода
    \item Точки разрыва второго рода
\end{enumerate}

\textbf{Теоремы:}
\begin{enumerate}
    \item Арифметические свойства непрерывных функций
    \item Непрерывность композиции непрерывных функций
\end{enumerate}


\section{}
\begin{st}
    Глобальные свойства непрерывных функций.
\end{st}

\textbf{Определения:}
\begin{enumerate}
    \item Непрерывность на множестве
\end{enumerate}

\textbf{Теоремы:}
\begin{enumerate}
    \item Первая теорема Вейерштрасса
    \item Вторая теорема Вейерштрасса
    \item Теорема о промежуточном значении функции
\end{enumerate}


\section{}
\begin{st}
    Теорема о разрывах монотонной функции. Теорема об обратной функции к непрерывной и монотонной.
\end{st}

\textbf{Теоремы:}
\begin{enumerate}
    \item Теорема о том, что у монотонной функции бывают разрывы только 1 рода
    \item Следствие для функции, которая определена на интервале
    \item Утверждение о том, что у монотонной функции разрывов не более чем счётное множество
    \item Теорема об обратной функции к непрерывной и монотонной
\end{enumerate}


\section{}
\begin{st}
    Равномерно непрерывные функции, теорема Кантора.
\end{st}

\textbf{Определения:}
\begin{enumerate}
    \item Равномерная непрерывность
\end{enumerate}

\textbf{Теоремы:}
\begin{enumerate}
    \item Теорема Кантора
\end{enumerate}


\section{}
\begin{st}
    Построение показательной функции. Логарифм, степенная функция, синус. Замечательные пределы.
\end{st}

\textbf{Теоремы:}
\begin{enumerate}
    \item Весь билет одна большая теорема
\end{enumerate}


\section{}
\begin{st}
    Производная функции, производная по множеству. Производная суммы, произведения и отношения функций.
\end{st}

\textbf{Определения:}
\begin{enumerate}
    \item Производная функции
    \item Производная функции по множеству
\end{enumerate}

\textbf{Теоремы:}
\begin{enumerate}
    \item Теорема о том, что если существует производная в точке, то функция непрерывна в данной точке
    \item Арифметические свойства производных
\end{enumerate}


\section{}
\begin{st}
    Производная композиции функций, производная обратной функции. Таблица производных.
\end{st}

\textbf{Теоремы:}
\begin{enumerate}
    \item Производная композиции функций
    \item Производная обратной функции
    \item Вывод некоторых формул из таблицы производных
\end{enumerate}


\section{}
\begin{st}
    Дифференцируемость функций, первый дифференциал. Связь между дифференцируемостью и существованием производной.
\end{st}

\textbf{Определения:}
\begin{enumerate}
    \item Полное приращение функции
    \item Дифференцируемая функция
    \item Первый дифференциал
\end{enumerate}

\textbf{Теоремы:}
\begin{enumerate}
    \item Связь между дифференцируемостью и существованием производной
\end{enumerate}


\section{}
\begin{st}
    Полукасательные и касательная к графику функции. Геометрический смысл первого дифференциала. Инвариантность первого дифференциала и неинвариантность производной.
\end{st}

\textbf{Определения:}
\begin{enumerate}
    \item Предельное положение семейства лучей
    \item Полукасательные
    \item Касательная
    \item Левая и правая производные
    \item Уравнение касательной
    \item Геометрический смысл первого дифференциала
    \item Инвариантность первого дифференциала и неинвариантность производной
\end{enumerate}


\section{}
\begin{st}
    Старшие производные. Старшие дифференциалы. Неинвариантность второго дифференциала.
\end{st}

\textbf{Определения:}
\begin{enumerate}
    \item Вторая производная
    \item Второй дифференциал
    \item n-ая производная
    \item Неинвариантность второго дифференциала
\end{enumerate}


\section{}
\begin{st}
    Теорема Ферма, необходимый признак локального экстремума, теорема Ролля. Формула Лагранжа и следствие из неё.
\end{st}

\textbf{Определения:}
\begin{enumerate}
    \item Точки экстремума
\end{enumerate}

\textbf{Теоремы:}
\begin{enumerate}
    \item Теорема Ферма и следствие из неё
    \item Необходимое условие существования экстремума
    \item Теорема Ролля
    \item Формула Лагранжа
    \item Следствие формулы Лагранжа
\end{enumerate}


\section{}
\begin{st}
    Формула Коши. Связь монотонности функции и знака производной.
\end{st}

\textbf{Теоремы:}
\begin{enumerate}
    \item Формула Коши
    \item Связь монотонности функции и знака производной
\end{enumerate}


\section{}
\begin{st}
    Отсутствие у производной дифференцируемой функции устранимых разрывов и разрывов первого рода.Теорема Дарбу о промежуточных значениях производной.
\end{st}

\textbf{Теоремы:}
\begin{enumerate}
    \item Отсутствие у производной дифференцируемой функции устранимых разрывов и разрывов первого рода
    \item Теорема Дарбу о промежуточных значениях производной
\end{enumerate}


\section{}
\begin{st}
    Правила Лопиталя.
\end{st}

\textbf{Теоремы:}
\begin{enumerate}
    \item Правила Лопиталя
\end{enumerate}


\section{}
\begin{st}
    Формула Тейлора. Остаточный член в форме Пеано и в общей форме. Остаточный член в форме Лагранжа.
\end{st}

\textbf{Определения:}
\begin{enumerate}
    \item Формула Тейлора
\end{enumerate}

\textbf{Теоремы:}
\begin{enumerate}
    \item Формула Тейлора с остаточным членом в форме Пеано
    \item Остаточный член в общей форме
    \item Остаточный член в форме Лагранжа
\end{enumerate}


\section{}
\begin{st}
    Формула Тейлора для основных элементарных функций. Достаточные условия локального экстремума. Общая схема поиска глобального экстремума функции на отрезке. Асимптоты.
\end{st}

\textbf{Определения:}
\begin{enumerate}
    \item Формулы Тейлора для основных элементарных функций
    \item Вертикальная асимптота
    \item Наклонная асимптота
\end{enumerate}

\textbf{Теоремы:}
\begin{enumerate}
    \item Достаточные условия локального экстремума
    \item Общая схема поиска глобального экстремума функции на отрезке
    \item Теорема о наклонной асимптоте
\end{enumerate}


\section{}
\begin{st}
    Выпуклые функции, достаточное условие выпуклости. Теорема о касательной к графику выпуклой функции. Точки перегиба. Необходимое и достаточные условия наличия точки перегиба. Неравенство Йенсена и следствие из него.
\end{st}

\textbf{Определения:}
\begin{enumerate}
    \item Выпуклая вверх (вниз) функция
    \item Точка перегиба
\end{enumerate}

\textbf{Теоремы:}
\begin{enumerate}
    \item Достаточное условие выпуклости
    \item Теорема о касательной к графику функции
    \item Необходимое и достаточные условия наличия точки перегиба
    \item Неравенство Йенсена
    \item Неравенство между средним арифметическим и средним геометрическим
    \item Неравенство Юнга
\end{enumerate}


\textbf{Итог: 122 определения; 96 теорем.}




\titleformat{\section}{\LARGE \bfseries}{}{1em}{}
\titleformat{\subsection}{\Large\bfseries}{}{1em}{}





% \section{Возможные задачи}

% \subsection{Теория множеств}
% \begin{enumerate}
%     \item $A \cap (B \cup C) = (A \cap B) \cup (A \cap C)$.
%     \item $A \cup (B \cup C) = (A \cup B) \cap (A \cup C)$.
%     \item $A \setminus (B \cup C) = (A \setminus B) \cap (A \setminus C)$.
%     \item $A \setminus (A \setminus B) = A \cap B$.
%     \item $A \setminus B = A \setminus (A \cap B)$.
%     \item $A \cap (B \setminus C) = (A \cap B) \setminus (A \cap C) = (A \cap B) \setminus C$.
%     \item $(A \setminus B) \ C = (A \setminus C) \setminus (B \setminus C)$.
%     \item $A \setminus (B \setminus A) = A$.
% \end{enumerate}

% \subsection{Вещественные числа}
% \begin{enumerate}
%     \item Доказать что $\sqrt{2}$ или, $\sqrt{5}$, или $\sqrt{7}$, $\sqrt[3]{2}$, $\sqrt[3]{3}$, $\sqrt{2} + \sqrt{3}$, $\sqrt{2} + \sqrt[3]{2}$, $\sqrt{2} + \sqrt{3} + \sqrt{5}$ иррациональны.
%     \item Доказать, что иррациональных чисел бесконечно много.
%     \item Приближённо вычислить $\sqrt[3]{2}$ и $\sqrt[3]{5}$ с ошибкой, не превышающей $10^{-2}$.
%     \item Представить в виде бесконечной периодической дроби $\frac{5}{13}$.
%     \item Представить в виде обыкновенной дроби $0,(194)$ или $0,27(194)$.
%     \item Найти $\sup[2;4)$, $\sup[2;4]$ , $\inf{ \frac{n}{n^2+1}, \ n \in \N}$, $\sup{\frac{n}{n^2+1}, n \in \N}$.
%     \item Найти $\underset{n \in \N}{\bigcap} \left[ 1 - \frac{1}{n}; 3 + \frac{1}{n} \right]$,
%     $\underset{n \in \N}{\bigcup} \left[ 1 - \frac{1}{n}; 3 + \frac{1}{n} \right]$,
%     $\underset{n \in \N}{\bigcup} \left[ 1 + \frac{1}{n}; 4 - \frac{1}{n} \right]$,
%     $\underset{n \in \N}{\bigcap} \left( 1 - \frac{1}{n}; 3 + \frac{1}{n} \right)$.
% \end{enumerate}

% \subsection{Топология вещественной прямой}

% \subsection{Числовые последовательности}

% \subsection{Предел функции}
% \begin{enumerate}
%     \item Доказать по определению $\liminf{\frac{n}{n^2+1}} = 0$.
%     \item Доказать $\liminf{\sqrt[n]{a}} = 1$.
%     \item Доказать $\liminf{\sqrt[n]{n}} = 1$.
%     \item Доказать монотонность и ограниченность последовательности $a_n = \left(1 - \frac{1}{2}\right) \cdot \left(1 - \frac{1}{4}\right) \cdot \ldots \cdot \left(1 - \frac{1}{2^n}\right), \ n \in \N$.
% \end{enumerate}

% \subsection{Непрерывность}

% \subsection{Дифференциальное исчисление}

% \subsection{Прочее}
% \begin{enumerate}
%     \item Найти обратные функции $y = 5x - 7$; $y = 1 + \frac{1}{x}$; $y = x^5$.
%     \item Найти $\arcsin{(\sin{3})}$; $\arctg{(\tg{3})}$; $\arcsin{(\sin{5})}$; $\arccos{(\cos{5})}$; $\arctg{(\tg{3})}$.
% \end{enumerate}



\end{document}